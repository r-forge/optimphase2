\HeaderA{plot.OptimDes}{Plot efficiency of optimal two-stage designs as a function of
total sample size}{plot.OptimDes}
\keyword{hplot}{plot.OptimDes}
\keyword{optimize}{plot.OptimDes}
\begin{Description}\relax
Output from function \code{OptimDes} is used to display the ETSL and
EDA for a two-stage design relative to a single-stage design as a
function of the combined stage 1 and 2 sample size.
\end{Description}
\begin{Usage}
\begin{verbatim}
## S3 method for class 'OptimDes':
plot(x, xscale = "t", l.type = 1:4, l.col =
c("blue", "green", "red"), CMadj=F, ...)
\end{verbatim}
\end{Usage}
\begin{Arguments}
\begin{ldescription}
\item[\code{x}] Output from function \code{OptimDes}.
\item[\code{xscale}] Scale of the x-axis. \code{"t"} for combined stage 1
and 2 study length.  \code{"n"} for combined sample size. Default is \code{t}.
\item[\code{l.type}] Line types for the plot. Default is \code{1-4}.
\item[\code{l.col}] Line colors for the plot. Default is \code{"blue"} for
\code{ETSL}, \code{"green"} for \code{EDA} and \code{"red"} for
\code{t1}.
\item[\code{CMadj}] If true, the sample sizes and times are adjusted by
the ratio of the exact binomial to asymptotic normal sample size for
the single stage design, as in Case and Morgan 2003.  Proportional
adjustment of times and samplesizes are made even if the accrual rates
are not constant.  Default is
false.  

\item[\code{...}] Additional graphical parameters passed to function \code{plot}.
\end{ldescription}
\end{Arguments}
\begin{Details}\relax
The plot displays the tradeoff between ETSL and EDA  as a function of
the combined sample size.  The robustness of the optimal two-stage  design to
deviations from the target sample size can be explored.  The plots
often suggest compromise designs achieving near-optimal results for
both EDA and ETSL.  Test boundary values (C1, C2), and numerical
values of other design parameters, can be obtained for a design
selected from the plots using function \code{\LinkA{np.OptimDes}{np.OptimDes}}.

The plot also includes the time of the interim analysis (\code{t1}) as a ratio to
the time for a corresonding single-stage analysis.
\end{Details}
\begin{Author}\relax
Bo Huang \email{<huang@stat.wisc.edu>} and Neal Thomas
\end{Author}
\begin{References}\relax
Case M. D. and Morgan T. M. (2003) Design of Phase II cancer trials
evaluating survival probabilities. \emph{BMC Medical Research
Methodology}, \bold{3}, 7.

Lin D. Y., Shen L., Ying Z. and Breslow N. E. (1996) Group seqential
designs for monitoring survival probabilities. \emph{Biometrics},
\bold{52}, 1033--1042.

Simon R. (1989) Optimal two-stage designs for phase II clinical
trials. \emph{Controlled Clinical Trials}, \bold{10}, 1--10.
\end{References}
\begin{SeeAlso}\relax
\code{\LinkA{print.OptimDes}{print.OptimDes}}, \code{\LinkA{OptimDes}{OptimDes}}
\end{SeeAlso}

