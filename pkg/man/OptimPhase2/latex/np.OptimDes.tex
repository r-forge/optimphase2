\HeaderA{np.OptimDes}{Optimal Two-Stage Design with User-specified Combined Sample Size}{np.OptimDes}
\keyword{design}{np.OptimDes}
\keyword{optimize}{np.OptimDes}
\begin{Description}\relax
Construct a two-stage design  for a time to event endpoint with possible stopping for futility
after the end of Stage 1.  The design  minimizes either the expected
duration of accrual (EDA) or the expected total study length
(ETSL).  The combined sample size for both stages is
pre-specifed by the user.
\end{Description}
\begin{Usage}
\begin{verbatim}

np.OptimDes(
            B.init, m.init, alpha, beta, param, x, n = NULL, pn = NULL,
           pt = NULL, target = c("EDA", "ETSL"), recover=TRUE,
           control = OptimDesControl(), CMadj=F, ...)
\end{verbatim}
\end{Usage}
\begin{Arguments}
\begin{ldescription}
\item[\code{B.init}] A vector of user-specified time points (B1, ..., Bb) that determine a
set of time intervals with uniform accrual.
\item[\code{m.init}] The projected
number of patients that can be accrued within the time intervals determined by \code{B.init}.
\item[\code{alpha}] Type I error.
\item[\code{beta}] Type II error.
\item[\code{param}] A four-element vector containing: the shape parameter of the Weibull distribution under the
null hypothesis, the scale parameter of the Weibull distribution under the
null hypothesis, the shape parameter of the Weibull distribution under the
alternative hypothesis, the scale parameter of the Weibull distribution under the
alternative hypothesis.
\item[\code{x}] Survival time of interest (e.g., 1 year).
\item[\code{n}] User-specified combined sample size for both stages. 
\item[\code{pn}] Combined sample size for both stages specified by the
ratio of the targetted two-stage sample size to the correponding
sample size for a single-stage design.
\item[\code{pt}] Combined sample size for both stages specified by the
ratio of the targetted two-stage study length to the correponding
study length for a single-stage design.
\item[\code{target}] When the expected duration of
accrual (EDA) is to be minimized, \code{target="EDA"}. When the
expected total study length is to be minimized,
\code{target="ETSL"}.
\item[\code{recover}] The Simon Two-Stage design recovers alpha from the Type I errors
that would have occured if the study had not been stopped early for
futility.  This recovery allows for a slightly lower boundary for the
final test, thereby increasing power.  The default with partial data
at the interim analysis is to recover the alpha as with the standard
Simon design.  When recover=F, the optimal design is found assuming
that the boundary for the final test will not be adjusted downward
ignoring this small potential gain to ensure protection of the alpha
level even when the accrual rates and survivor functions are mis-specified. 

\item[\code{control}] An optional list of control settings.  See
\code{\LinkA{OptimDesControl}{OptimDesControl}}
for the parameters that can be set and their default values.
\item[\code{CMadj}] If CMadj=T, the ratios in \code{pn} and \code{pt} use
the exact binomial sample size in the denominator. The \code{CMadj}
has no effect when \code{n} is specified.  The default is \code{CMadj=F}.
\item[\code{...}] No additional optional parameters are currently implemented
\end{ldescription}
\end{Arguments}
\begin{Details}\relax
Plots (\code{\LinkA{plot.OptimDes}{plot.OptimDes}}) based on the ouput of
\code{\LinkA{OptimDes}{OptimDes}} can be used to find compromise designs based on
different combined sample sizes (stages 1 and 2) with
near optimal values for both ETSL and EDA.  \code{np.OptimDes} can be
used to compute ETSL, EDA, and the other design parameters for any
specified total sample size.

The targetted combined sample size must be specified by one of
three equivalent approaches: \code{n}, \code{pn}, and \code{pt}.
\end{Details}
\begin{Value}
A list of class \code{OptimDes} with the same output as function \code{OptimDes}.
\end{Value}
\begin{Author}\relax
Bo Huang \email{<huang@stat.wisc.edu>} and Neal Thomas
\end{Author}
\begin{References}\relax
Case M. D. and Morgan T. M. (2003) Design of Phase II cancer trials
evaluating survival probabilities. \emph{BMC Medical Research
Methodology}, \bold{3}, 7.

Lin D. Y., Shen L., Ying Z. and Breslow N. E. (1996) Group seqential
designs for monitoring survival probabilities. \emph{Biometrics},
\bold{52}, 1033--1042.

Simon R. (1989) Optimal two-stage designs for phase II clinical
trials. \emph{Controlled Clinical Trials}, \bold{10}, 1--10.
\end{References}
\begin{SeeAlso}\relax
\code{\LinkA{OptimDes}{OptimDes}}, \code{\LinkA{plot.OptimDes}{plot.OptimDes}}
\end{SeeAlso}
\begin{Examples}
\begin{ExampleCode}
B.init <- c(1, 2, 3, 4, 5)
m.init <- c(15, 20, 25, 20, 15)
alpha <- 0.05
beta <- 0.1
param <- c(1, 1.09, 2, 1.40)
x <- 1

# H0: S0=0.40 H1: S1=0.60

np.OptimDes(B.init,m.init,alpha,beta,param,x,pt=1.1,target="ETSL")
\end{ExampleCode}
\end{Examples}

