\HeaderA{FixDes}{Construct a Single-stage Design for a Time to Event
Endpoint Versus a Known Control Standard}{FixDes}
\keyword{design}{FixDes}
\begin{Description}\relax
Find the sample size, duration of accrual, and test boundary for a single-stage
design with time to event endpoint versus a known control standard.
Testing is one-sided based on the Kaplan-Meier estimator at a
user-specified time point.
\end{Description}
\begin{Usage}
\begin{verbatim}
FixDes(B.init, m.init, alpha, beta, param, x)
\end{verbatim}
\end{Usage}
\begin{Arguments}
\begin{ldescription}
\item[\code{B.init}] A vector of user-specified time points (B1, ..., Bb) that determine a
set of time intervals with uniform accrual.
\item[\code{m.init}] The projected
number of patients that can be accrued within the time intervals determined by \code{B.init}.
\item[\code{alpha}] Type I error.
\item[\code{beta}] Type II error.
\item[\code{param}] A four-element vector containing: the shape parameter of the Weibull distribution under the
null hypothesis, the scale parameter of the Weibull distribution under the
null hypothesis, the shape parameter of the Weibull distribution under the
alternative hypothesis, the scale parameter of the Weibull distribution under the
alternative hypothesis.
\item[\code{x}] Survival time of interest (e.g., 1 year).
\end{ldescription}
\end{Arguments}
\begin{Details}\relax
Estimation is based on the Kaplan-Meier or Nelson-Aalen estimators
evaluated at a target time (e.g., 1 year).  The event rates at the
target
time are computed from Weibull distributions assumed for the treated
and (assumed known) control distributions, as is done in function \code{\LinkA{OptimDes}{OptimDes}}.
The design depends only on the event rates at the target time (except
for small changes due to rounding with different survivor functions).
The duration of accrual depends on the projected maximum accrual rates.
\end{Details}
\begin{Value}
A list with components:
\begin{ldescription}
\item[\code{n0}] Fixed design sample size
\item[\code{DA}] Duration of accrual
\item[\code{SL}] Total study length (\code{time, DA+x}).
\item[\code{n0E}] n0 based on exact binomial test
\item[\code{DAE}] DA based on exact binomial test
\item[\code{SLE}] SL based on exact binomial test
\item[\code{C}] Rejection cutpoint for the test statistic.
\end{ldescription}
\end{Value}
\begin{Note}\relax
The single-stage sample size is used as the starting value for evaluating the
optimal \code{n} for a two-stage design in \code{\LinkA{OptimDes}{OptimDes}}.
\end{Note}
\begin{Author}\relax
Bo Huang \email{<huang@stat.wisc.edu>} and Neal Thomas
\end{Author}
\begin{References}\relax
Case M. D. and Morgan T. M. (2003) Design of Phase II cancer trials
evaluating survival probabilities. \emph{BMC Medical Research
Methodology}, \bold{3}, 7.

Lin D. Y., Shen L., Ying Z. and Breslow N. E. (1996) Group seqential
designs for monitoring survival probabilities. \emph{Biometrics},
\bold{52}, 1033--1042.

Simon R. (1989) Optimal two-stage designs for phase II clinical
trials. \emph{Controlled Clinical Trials}, \bold{10}, 1--10.
\end{References}
\begin{SeeAlso}\relax
\code{\LinkA{OptimDes}{OptimDes}}, \code{\LinkA{Test2stage}{Test2stage}}, \code{\LinkA{SimDes}{SimDes}}
\end{SeeAlso}
\begin{Examples}
\begin{ExampleCode}
B.init <- c(1, 2, 3, 4, 5)
m.init <- c(15, 20, 25, 20, 15)
alpha <- 0.05
beta <- 0.1
param <- c(1, 1.09, 2, 1.40)
x <- 1

# H0: S0=0.40 H1: S1=0.60

FixDes(B.init, m.init, alpha, beta, param, x)

\end{ExampleCode}
\end{Examples}

