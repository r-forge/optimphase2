\HeaderA{Test2stage}{Statistical test for two-stage designs from function OptimDes}{Test2stage}
\keyword{htest}{Test2stage}
\keyword{optimize}{Test2stage}
\begin{Description}\relax
This function performs the hypothesis tests for the two-stage designs
with time to event endpoint from \code{\LinkA{OptimDes}{OptimDes}}.
\end{Description}
\begin{Usage}
\begin{verbatim}
Test2stage(Y1, T1, Y2 = NULL, T2 = NULL, p0, x, C1, C2, t1, MTSL =
NULL, printTest = TRUE, cen1=rep(1,length(T1)),cen2=rep(1,length(T2)))
\end{verbatim}
\end{Usage}
\begin{Arguments}
\begin{ldescription}
\item[\code{Y1}] A vector containing the study start times (measured from
the beginning of the study) of first-stage patients.
\item[\code{T1}] A vector containing the event times of first-stage patients.
\item[\code{Y2}] A vector containing the study start times (measured from
the beginning of the study) of second-stage patients.
\item[\code{T2}] A vector containing the event times of second-stage
patients.
\item[\code{p0}] The event rate under the null hypothesis.
\item[\code{x}] Survival time of interest (e.g., 1 year).
\item[\code{C1}] The two-stage study is terminated for futility after the
first stage if the Z-statistic is < C1
\item[\code{C2}] Reject the null hypothesis for the one-sided alternative
hypothesis at the second stage if the Z-statistic>C2.
\item[\code{t1}] Time for the interim analysis (actual time, which may
differ from the planned time).
\item[\code{MTSL}] Maximum total study length, (actual, not planned)
\item[\code{printTest}] If TRUE (default), the result of the test and the
interim decision is printed.
\item[\code{cen1}] The times in \code{T1} are regarded as events unless
they are set to censored by setting the corresponding value in
\code{cen1} to zero.
\item[\code{cen2}] The times in \code{T2} are regarded as events unless
they are set to censored by setting the corresponding value in
\code{cen1} to zero.
\end{ldescription}
\end{Arguments}
\begin{Details}\relax
The hypothesis tests are performed in two stages as described in Case
and Morgan (2003).

Stage 1. Accrue patients between time 0 and time \code{t1}. Each
patient will be followed until failure, or for \code{x} years or until
time \code{t1}, whichever is less. Calculate the normalized interim test
statistic \code{Z1}. If \code{Z1<C1}, stop the study for futility;
otherwise, continue to the next stage.

Stage 2. Accrue patients between \code{t1} and \code{MDA}. Follow all
patients until failure or for \code{x} years, then calculate the
normalized final test statistic \code{Z2}, and reject H0 if
\code{Z2>C2}.

The test statistic is based on the Nelson-Aalen estimator of the
cumulative hazard function.
\end{Details}
\begin{Value}
A vector containing results for the interim analysis if \code{T2} and
\code{Y2} are NULL, and the 
final analysis if \code{T2} or \code{Y2} are non-NULL:

\begin{ldescription}
\item[\code{z}] The test statistic
\item[\code{se}] Standard error of the cummulative hazard (not log
cummulative hazard) at time \code{x}.
\item[\code{cumL}] Cummulative hazard estimator at time \code{x}.
\end{ldescription}
\end{Value}
\begin{Author}\relax
Bo Huang \email{<huang@stat.wisc.edu>} and Neal Thomas
\end{Author}
\begin{References}\relax
Case M. D. and Morgan T. M. (2003) Design of Phase II cancer trials
evaluating survival probabilities. \emph{BMC Medical Research
Methodology}, \bold{3}, 7.

Lin D. Y., Shen L., Ying Z. and Breslow N. E. (1996) Group seqential
designs for monitoring survival probabilities. \emph{Biometrics},
\bold{52}, 1033--1042.

Simon R. (1989) Optimal two-stage designs for phase II clinical
trials. \emph{Controlled Clinical Trials}, \bold{10}, 1--10.
\end{References}
\begin{SeeAlso}\relax
\code{\LinkA{OptimDes}{OptimDes}}, \code{\LinkA{SimDes}{SimDes}}
\end{SeeAlso}
\begin{Examples}
\begin{ExampleCode}
B.init <- c(1, 2, 3, 4, 5)
m.init <- c(15, 20, 25, 20, 15)
alpha <- 0.05
beta <- 0.1
param <- c(1, 1.09, 2, 1.40)
x <- 1

# H0: S0=0.40 H1: S1=0.60

shape0 <- param[1]
scale0 <- param[2]
shape1 <- param[3]
scale1 <- param[4]

object1 <- OptimDes(B.init,m.init,alpha,beta,param,x,target="EDA")
n <- object1$n[2]
t1 <- object1$stageTime[1]
C1 <- object1$boundary[1]
C2 <- object1$boundary[2]
b <- length(B.init)
l <- rank(c(cumsum(m.init),n),ties.method="min")[b+1]
mda <- ifelse(l>1,B.init[l-1]+(B.init[l]-B.init[l-1])*(n-sum(m.init[1:(l-1)]))/m.init[l],B.init[l]*(n/m.init[l]))

### set up values to create a stepwise uniform distribution for accrual
B <- B.init[1:l]
B[l] <- mda
xv <- c(0,B)
M <- m.init[1:l]
M[l] <- ifelse(l>1,n-sum(m.init[1:(l-1)]),n)
yv <- c(0,M/(diff(xv)*n),0)

# density function of accrual 
dens.Y <- stepfun(xv,yv,f=1,right=TRUE)
# pool of time points to be simulated from
t.Y <- seq(0,mda,by=0.01)

# simulate study times of length n
sample.Y <- sample(t.Y,n,replace=TRUE,prob=dens.Y(t.Y))

# simulate failure times of length n under the alternative hypothesis
sample.T <- rweibull(n,shape=shape1,scale=scale1)

Y1 <- sample.Y[sample.Y<=t1]
T1 <- sample.T[sample.Y<=t1]
Y2 <- sample.Y[sample.Y>t1]
T2 <- sample.T[sample.Y>t1]

# event rate under null hypothesis
p0<-pweibull(x,shape=shape0,scale=scale0)

# interim analysis
Test2stage(Y1, T1, Y2 = NULL, T2 = NULL, p0, x, C1, C2, t1, MTSL = NULL)

# final analysis if the study continues
Test2stage(Y1, T1, Y2, T2, p0, x, C1, C2, t1, mda+x)

# simulate failure times of length n under the null hypothesis
sample.T <- rweibull(n,shape=shape0,scale=scale0)

Y1 <- sample.Y[sample.Y<=t1]
T1 <- sample.T[sample.Y<=t1]
Y2 <- sample.Y[sample.Y>t1]
T2 <- sample.T[sample.Y>t1]

# interim analysis
Test2stage(Y1, T1, Y2 = NULL, T2 = NULL, p0, x, C1, C2, t1, MTSL = NULL)

# final analysis if the study continues
Test2stage(Y1, T1, Y2, T2, p0, x, C1, C2, t1, mda+x)

\end{ExampleCode}
\end{Examples}

