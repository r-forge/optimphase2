% \VignetteIndexEntry{OptimPhase2 Manual}
% \VignetteDepends{OptimPhase2}
% \VignettePackage{OptimPhase2}
\documentclass[12pt]{article}



\usepackage{amsmath}
\usepackage[pdftex]{graphicx}
\usepackage{color}
\usepackage{xspace}
\usepackage{fancyvrb}
\usepackage{fancyhdr}
\usepackage{lastpage}
\usepackage{algorithm2e}
\usepackage[
         colorlinks=true,
         linkcolor=blue,
         citecolor=blue,
         urlcolor=blue]
         {hyperref}
\usepackage{Sweave}

\newcommand{\bld}[1]{\mbox{\boldmath $#1$}}
\newcommand{\shell}[1]{\mbox{$#1$}}
\renewcommand{\vec}[1]{\mbox{\bf {#1}}}

%\newcommand{\ReallySmallSpacing}{\renewcommand{\baselinestretch}{.6}\Large\normalsize}
\renewcommand{\baselinestretch}{1.3}

\newcommand{\halfs}{\frac{1}{2}}

%\setlength{\oddsidemargin}{-.25 truein}
%\setlength{\evensidemargin}{0truein}
%\setlength{\topmargin}{-0.2truein}
%\setlength{\textwidth}{7 truein}
%\setlength{\textheight}{8.5 truein}
%\setlength{\parindent}{0.20truein}
%\setlength{\parskip}{0.10truein}



%%%%%%%%%%%%%%%%%%%%%%%%%%%%%%%%%%%%%%%%%%%%%%%%%%%%%%%%%%%%%%%%%%
\pagestyle{fancy}
\lhead{}
\chead{The \texttt{OptimPhase2} Package}
\rhead{}
\lfoot{}


\renewcommand{\headrulewidth}{1pt}
\renewcommand{\footrulewidth}{1pt}
%%%%%%%%%%%%%%%%%%%%%%%%%%%%%%%%%%%%%%%%%%%%%%%%%%%%%%%%%%%%%%%%%%

\begin{document}

\title{The \texttt{OptimPhase2} Package}
\author{Bo Huang and Neal Thomas\\Pfizer Inc. }
\maketitle

\tableofcontents
%\listoftables
%\listoffigures

%\addcontentsline{toc}{section}{Introduction}
\section{Introduction}
Phase II oncology trials are undertaken to  assess the activity of a new treatment with activity
most frequently defined in terms of tumor response. With some of the new targeted therapies, it may
not be appropriate to use tumor shrinkage to evaluate activity. Instead, the primary endpoint is
time to death, time to progression, or some other endpoints that must be evaluated over a longer
time period.

Phase II trials are often designed with an interim analysis so they can be stopped early if a drug
is ineffective. However, when the primary endpoint requires a longer observation period, interim
analyses are challenging because of incomplete follow-up for some patients at the time of the
interim analysis.  Standard designs such as Simon (1989) require suspension of accrual while
patient follow-up is completed.  Case and Morgan (2003) presented a two-stage design for a phase II
oncology trial with a long-term endpoint that does not suspend accrual while the interim analysis
is conducted.  They proposed to use the Kaplan-Meier or Nelson-Aalen estimators of the event
probability, using methods like those in Lin et al (1996). Estimation at the time of the interim
analysis includes patients with partial follow-up without necessitating trial suspension, as also
proposed by Jennison and Turnbull (2000).  The
design minimizes either the expected sample size, expected duration of accrual, or the expected
total study length under the hypothesis that the drug is ineffective. The null hypothesis for the
new design is an (assumed) known event-free rate within a specified time, which has been judged to
represent ineffective treatment.  This is similar to  the hypothesis in the Simon design, but with
much longer specified times for events to occur.  Schaid, Wieand, and Therneau (1990) proposed a
similar design using the log rank statistic, which also incorporates patients with incomplete
follow-up;  the log rank statistic is not evaluated here, but could be inlcuded as a future
software option.


Both Lin et al (1996) and Case and Morgan (2003) assume a constant accrual rate throughout the
trial, which is not typical in practice. We further investigate the design properties by
generalizing the accrual distribution to have different accrual rates in user-specified intervals.
As noted in Case and Morgan (2003), when only partial follow-up data are available, the level of
the testing procedure can depend on the assumed accrual distribution and the assumed time-to-event
distribution under the null hypothesis.  We evaluate an optimal design and corresponding analysis
that ensure the Type I error rate is below the target level.  The reduction in power or
corresponding increase in sample size necessary to achieve the conservative type I error rates is
also evaluated.

The theoretical derivations of the optimal designs specify a fixed time to end the first stage of
accrual and to conduct the interim analysis, with corresponding projected sample size. Case and
Morgan (2003) also evaluated a modified interim timing rule that ends the first stage when the
projected number of patients has been accrued regardless of the planned interim time. They showed
this rule has more robust statistical properties when the accrual rate is mis-specified.  We also
evaluate this interim timing rule and an additional rule based on the projected patient exposure at
the optimal interim time. This rule not only accounts for the number of patients actually accrued,
but also the length of time patients have been observed. All of the interim timing rules can be
easily applied in practice.

This R package {\it OptimPhase2} was created to generate the optimal designs and resulting analyses. The package
includes code to perform simulations to validate the theoretical calculations, some of which depend
on asymptotic approximations.  The package also has several options for evaluating a proposed
design under conditions that differ from those assumed when the design was created. All these will be
elaborated with examples in the subsequent sections.

\section{Generate the Optimal Designs}
\label{OptimDes}
\subsection{Optimal Design Functions}

\begin{center}
\begin{verbatim}
OptimDes(
B.init, m.init, alpha, beta, param, x, target = c("EDA", "ETSL","ES"),
recover=TRUE, control = OptimDesControl(),...)

np.OptimDes(
B.init, m.init, alpha, beta, param, x, n = NULL, pn = NULL,
pt = NULL, target = c("EDA", "ETSL","ES"), recover=TRUE,
control = OptimDesControl(), CMadj=F, ...)
\end{verbatim}
\end{center}

{\it OptimDes} finds a two-stage design for a time to event endpoint with potential stopping for
futility after the first stage. It implements the Case and Morgan (2003) generalization of the
Simon (1989) two-stage design for comparing a treatment to a known standard. The design minimizes
either the expected duration of accrual (EDA), expected sample size (ES), or the expected total
study length (ETSL).

The design calculations assume Weibull distributions for the event-free endpoint in the treated
group, and for the (assumed known, "Null") control distribution. The function {\it weibPmatch} (see
Section \ref{weibull}) can be used to select Weibull parameters that yield a target event-free rate
at a specified time. Estimation is based on the Kaplan-Meier or Nelson-Aalen estimators evaluated
at a target time (e.g., 1 year). The treatment and control distributions and the accrual
distribution affect power (and alpha level in some settings), see Case and Morgan (2003)).

Accrual rates are specified by the user. These rates can differ across time intervals specified by the user (this
generalizes the results in Case and Morgan). The accrual information is controlled by arguments {\it B.init} and
{\it m.init}.

{\it OptimDes} also has the capability to choose to recover or not recover the amount of $\alpha$ potentially saved
at the interim analysis (analogous to non-binding in group sequential designs).

{\bf Note:} Details of {\it OptimPhase2} and all subsequent package functions can be found on the help pages.

\subsection{Examples}
\label{optexample}

Assume the 1-year survival rate of a standard cancer therapy is 0.40 ($H_0$). An improvement to 0.60 would be
considered clinically significant ($H_1$). Assume the survival distributions have different shapes and scales under
null and the alternative, determined by the weibull parameters $(1, 1.09)$ under $H_0$ and $(2, 1.40)$ under $H_1$.
Type I error is 0.05. Type II error is 0.1. It is also assumed that the numbers of patients that can be enrolled in
the first 5 years are 15, 20, 25, 20 and 15 respectively.
\begin{Schunk}
\begin{Sinput}
> B.init <- c(1, 2, 3, 4, 5)
> m.init <- c(15, 20, 25, 20, 15)
> alpha <- 0.05
> beta <- 0.1
> param <- c(1, 1.09, 2, 1.4)
> x <- 1
\end{Sinput}
\end{Schunk}

The optimal design {\it object1} minimizing the expected duration of accrual (EDA) after
implementing {\it OptimDes} can then be obtained.

\begin{Schunk}
\begin{Sinput}
> object1 <- OptimDes(B.init, m.init, alpha, beta, param, x, target = "EDA")
> print(object1)
\end{Sinput}
\begin{Soutput}
             Optimal Design Results

    H0: S0= 0.4 H1: S1= 0.6

    Type I error(1-sided upper): 0.05 type II error: 0.1
    Recover alpha: TRUE
    Event-free time of interest: 1

                  target: EDA
          EDA          ETSL            ES
        2.777         3.243        53.253

          Sample Size at Each Stage
              n1           n.last
              39               70

          Study time at Each Stage
               t1              MTSL
            2.145             4.500

    Projected patient exposure at interim analysis:  28.22

    Proportion of the total information at the interim analysis:
           Under NuLL     Under Alternative
                0.361                 0.330


          Hypothesis Test Boundaries
               C1                C2
            0.085             1.601


    Approximate Rates Corresponding to Test Boundaries*

          Event-free rate for C1:  0.412
          Event-free rate for C2:  0.5


    Single-stage Design (Asymptotic normal calculation)
      Single stage N                   DA                   SL
                64.0                  3.2                  4.2


 *Note:  Rates corresponding to test boundaries are a function
 of the non-parametric SE computed at the time of the analyses.
 The approximate rates are based on the asymptotic SE computed
 under the null and alternative hypotheses.
\end{Soutput}
\end{Schunk}

A plot function {\it plot.OptimDes} is used to display the ETSL, ES and EDA for a two-stage design relative to a
single-stage design as a function of the combined stage 1 and 2 sample size. It demonstrates the tradeoff between
ETSL, EDA and ES as a function of the combined sample size. Robustness of the optimal two-stage design to deviations
from the target sample size can be explored. The plot often suggests a compromised design achieving near-optimal results
for both EDA and ETSL be a favorable design to the optimal one based on a single criteria.
Test boundary values $(C_1, C_2)$,
and numerical values of other design parameters, can be obtained for a design selected from the plot using function
{\it np.OptimDes}. Thus, {\it np.OptimDes} generates the optimal design when the total sample size is fixed.

Using the above case as an example, let's assume the ETSL is minimized this time. Then {\it object2} is the corresponding
optimal design with the optimal plot Figure \ref{obj2}. The optimal design is displayed as the sold circle on the plot.
If investigators believe a compromised design with maximum study length ratio = 1.1 ($pt=1.1$ in {\it np.OptimDes}) will
save some patients while still producing near-optimal results, $pt=1.1$ can be input into {\it np.OptimDes} and the
adjusted ''optimal" design can be created


\begin{figure}[htbp]
\begin{center}
\includegraphics[scale=0.6]{object2.pdf}
\caption{The optimality criteria displayed for a range of maximum sample sizes.  The criteria and
the maximum sample sizes are expressed as ratios relative to the corresponding value in a
single-stage fixed design.} \label{obj2}
\end{center}
\end{figure}

\begin{Schunk}
\begin{Sinput}
> object3 <- np.OptimDes(B.init, m.init, alpha, beta, param, x,
+     pt = 1.1, target = "ETSL")
> print(object3)
\end{Sinput}
\begin{Soutput}
             Optimal Design Results

    H0: S0= 0.4 H1: S1= 0.6

    Type I error(1-sided upper): 0.05 type II error: 0.1
    Recover alpha: TRUE
    Event-free time of interest: 1

                  target: ETSL
          EDA          ETSL            ES
        2.835         3.161        54.819

          Sample Size at Each Stage
              n1           n.last
              47               73

          Study time at Each Stage
               t1              MTSL
            2.441             4.650

    Projected patient exposure at interim analysis:  35.03

    Proportion of the total information at the interim analysis:
           Under NuLL     Under Alternative
                0.432                 0.400


          Hypothesis Test Boundaries
               C1                C2
            0.450             1.572


    Approximate Rates Corresponding to Test Boundaries*

          Event-free rate for C1:  0.454
          Event-free rate for C2:  0.496


    Single-stage Design (Asymptotic normal calculation)
      Single stage N                   DA                   SL
                64.0                  3.2                  4.2


 *Note:  Rates corresponding to test boundaries are a function
 of the non-parametric SE computed at the time of the analyses.
 The approximate rates are based on the asymptotic SE computed
 under the null and alternative hypotheses.
\end{Soutput}
\end{Schunk}

\section{Test Statistics and Decision Rules at Each Stage}
\label{test}
\begin{center}
\begin{verbatim}
Test2stage(
Y1, T1, Y2 = NULL, T2 = NULL, p0, x, C1, C2, t1, MTSL =
NULL, printTest = TRUE, cen1=rep(1,length(T1)),cen2=rep(1,length(T2)))
\end{verbatim}
\end{center}

The test statistic at the end of each stage is computed and compared to the decision boundaries $C_1$ and $C_2$.

\begin{itemize}
\item
Stage 1: Accrue $n_1$ patients between time $0$ and time $t_1$. Each patient is followed
until they have an event or successfully reach time $x$, or until study time $t_1$, whichever is
first. Calculate the normalized Z-statistic by {\it Test2stage}, and denote it by $Z_1(x;t_1)$. If
$Z_1(x;t_1)<C_1$, stop the study for futility; otherwise, continue to the next stage.  The probability of
stopping under the null hypothesis is approximated by $P_s=\Phi(C_1)$, where $\Phi$ is the standard normal
cummulative distribution function.  $n_1$ is  a random variable determined by $t_1$ and the accrual
distribution.
\item
Stage 2: Accrue $n_2$ additional patients between times $t_1$ and maximum duration of accrual ($MDA$). Follow all
patients (both stages) until they have an event or successfully reach time $x$, then calculate a
second $Z$ statistic at the end of maximum total study length ($MTSL$), denoted by $Z_2(x;MTSL)$, and
reject $H_0$ if $Z_2(x;MTSL)>C_2$.
\end{itemize}

For example, if at the end of Stage 1, the test statistic $Z_1=3.391$, $C_1=0.085$. Then {\it Test2stage} will return
\begin{center}
\begin{verbatim}
Z1 >= C1, continue to the second stage
\end{verbatim}
\end{center}


\section{Simulation Studies}
\label{SimDes}

\begin{center}
\begin{verbatim}
SimDes(
object,B.init,m.init,weib0,weib1,interimRule='e1',
sim.n=1000,e1conv=1/365,CMadj=F,attainI=1,attainT=1,
FixDes="F", Rseed)
\end{verbatim}
\end{center}

The {\it SimDes} function is a powerful function to simulate experiments to compare the true alpha
level and power of two-stage designs from function {\it OptimDes} with the targeted nominal values.
It can also be used to assess the performance of the optimal design under mis-specification of the
design parameters. For example, if the Weibull shape and scale parameters of the time to event
distributions are changed, if the accrual rates deviate from the projected ones, or if the interim
analysis is conducted differently from the planned one under the more realistic conditions. In
addition, the function has the option to determine the timing of the interim analysis by matching
the observed information to the expected time, number of patients or patient exposure
(interimRule=``t1", ``n1" or ``e1").

\subsection{Example 1: Optimal Settings}
Recall that in Section \ref{optexample} {\it object1} is the optimal design minimizing the EDA. Under the expected
parameter settings, 1000 simulations are conducted by matching the expected patient exposure at the interim
\begin{Schunk}
\begin{Sinput}
> (sim1 <- SimDes(object1))
\end{Sinput}
\begin{Soutput}
 The interimRule is  e1

alphaExact  alphaNorm powerExact  powerNorm        eda       etsl         es
 0.0380000  0.0630000  0.9290000  0.9470000  2.8231925  3.3675778 55.6160000
 pstopNull   pstopAlt       aveE  pinfoNull   pinfoAlt         n1         t1
 0.4730000  0.0230000 28.2249407  0.3433120  0.3295983 39.6350000  2.1605113
    phatKl     phatKh     phatRl     phatRh
 0.4081223  0.4081139  0.5035585  0.4893747
\end{Soutput}
\end{Schunk}

Details of the returned values can be found from the help pages. For instance, the estimated alpha level using
an exact test for the second stage test is 0.038.

\subsection{Example 2: Differed Accrual Rates}
Now suppose the actual numbers of patients that can be accrued in the first 5 years are different from the originally
planned for the optimal design ({\it m.init} below), then the results after 1000 simulated trials become
\begin{Schunk}
\begin{Sinput}
> (sim2 <- SimDes(object1, m.init = c(5, 5, 25, 25, 25)))
\end{Sinput}
\begin{Soutput}
 The interimRule is  e1

alphaExact  alphaNorm powerExact  powerNorm        eda       etsl         es
 0.0320000  0.0490000  0.9200000  0.9400000  3.7683860  4.2782850 55.3520000
 pstopNull   pstopAlt       aveE  pinfoNull   pinfoAlt         n1         t1
 0.5090000  0.0360000 28.2250430  0.3220282  0.2925826 41.2330000  3.2325549
    phatKl     phatKh     phatRl     phatRh
 0.4078247  0.4081419  0.5035585  0.4893747
\end{Soutput}
\end{Schunk}


\subsection{Example 3: Differed Interim Timing}
If the actual interim time or sample size (depending on {\it interimRule},{\it interimRule=``t1"} below)
is different from the originally
planned for the optimal design ({\it attainI}=0.8 below), then the results after 1000 simulated trials become
\begin{Schunk}
\begin{Sinput}
> (sim3 <- SimDes(object1, interimRule = "t1", attainI = 0.8))
\end{Sinput}
\begin{Soutput}
 The interimRule is  t1

alphaExact  alphaNorm powerExact  powerNorm        eda       etsl         es
 0.0320000  0.0440000  0.9040000  0.9150000  2.5621788  3.0886060 49.6330000
 pstopNull   pstopAlt       aveE  pinfoNull   pinfoAlt         n1         t1
 0.4980000  0.0530000 19.2944758  0.2064760  0.1932355 29.0550000  1.7162918
    phatKl     phatKh     phatRl     phatRh
 0.4091581  0.4143881  0.5035585  0.4893747
\end{Soutput}
\end{Schunk}
)

\section{Survival Curves Based on the Weibull Distribution}
\label{weibull}
\begin{center}
\begin{verbatim}
weibPmatch(x, p0, shape, scale)

weibull.plot(param, x, l.type = 1:3, l.col = c("blue", "red"), ...)
\end{verbatim}
\end{center}

{\it weibPmatch} and {\it weibull.plot} are used together to determine the shape and scale parameters of the
Weibull distrbution for the survival curves under $H_0$ and $H_1$. The Weibull distribution is flexible enough
to cover the majority of scenarios likely to encounter in practice.

{\it weibPmatch} determines the shape or
scale parameter of a Weibull distribution so it has event-free rate $P_0$ at time $x$. If the shape is specified,
the scale parameter is computed, and if the scale is specified, the shape parameter is computed.

{\it weibull.plot} then plots Weibull survival curves with differences at a target time highlighted from the parameters
computed from {\it weibPmatch}. Figure \ref{weib} is an example plot implementing the Weibull parameters input
to {\it OptimDes} to create {\it object1}.


\begin{figure}[htbp]
\begin{center}
\includegraphics[scale=0.6]{weib.pdf}
\caption{Survival curves under the Weibull distribution} \label{weib}
\end{center}
\end{figure}



\newpage

\addcontentsline{toc}{section}{Bibliography}
\begin{thebibliography}{}
\bibitem{agresti02} Agresti, A. (2002). Categorical Data Analysis. {\it Weiley}
\bibitem{brent73} Brent, R. (1973). Algorithms for minimization without derivatives. {\it Englewood Cliffs, Prentice Hall}
\bibitem{cm2003} Case L.D. and Morgan, T.M. (2003). Design of phase II cancer trials evaluating survival probabilities. {\it BMC Medical Research Methodology} {\bf 3}: 6
\bibitem{guide06} Food and Drug Administration (2006).  Guidance for Clinical Trial Sponsors:  Establishment and operation of clinical trial data monitoring committees.  {\it OMB Control No. 0910-0581}
\bibitem{jonker07} Jonker DJ et al. (2007).  Abstract No. LB-1, {\it American Association for Cancer Research}

\bibitem{km58} Kaplan, E.L. and Meier, P. (1958). Nonparametric estimation from incomplete observations. {\it Journal of American Statistical Association}. {\bf 53}: 457-481
\bibitem{lin96} Lin, D.Y., Shen, L., Ying, Z. and Breslow, N.E. (1996). Group sequential designs for monitoring survival probabilities. {\it Biometrics}. {\bf 52}: 1033-1042
\bibitem{machin08} Machin, D., Campbell, M., Tan, S.B. and Tan, S.H. (2008). Sample Size Tables for
Clinical Studies. {\it Blackwell}
\bibitem{nelson69} Nelson, W. (1969). Hazard plotting for incomplete failure data. {\it J Quality Technology}. {\bf 1}: 27-52
\bibitem{rao04} Rao S, Cunningham D, de Gramont A, et al (2004).  Phase III double-blind placebo-controlled study of farnesyl transferase
inhibitor R115777 in patients with refractory advanced colorectal cancer.
{\bf 22(19)}: 3950-3957
\bibitem{schaid90} Schaid, D., Wieand, S., and Therneau, T. (1990).  Optimal two-stage screening designs for survival comparisons. {\it Biometrika}.
{\bf 77}: 507-513
\bibitem{simon89} Simon, R. (1989). Optimal two-stage designs for phase II clinical trials. {\it Controlled Clinical Trials}. {\bf 10}: 1-10
\bibitem{vc07} Van Cutsem E, Peeters M, et.al. (2007).  Open-label phase III trial of panitumumab plus best
supportive care compared with best supportive care alone in patients with
chemotherapy-refractory metastatic colorectal cancer. {\it Journal of Clinical Oncology}. {\bf 1;25(13)}:1658-1664.
\end{thebibliography}

\end{document}
